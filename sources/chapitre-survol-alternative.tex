% === Cours de Java
% === Chapitre : Survol 

\subsection{Alternatives (survol)}

\imgfullh{../img/The_choice_by_vallo29.jpg}
{\centering\color{bluepigment}\Large\bf Alternatives}
{http://vallo29.deviantart.com/art/The-choice-150871274}

\begin{frame}[fragile]{Instructions de choix}
Le \emph{Si}
\begin{Java}
  if ( condition ) {
    instructions
  }
\end{Java}
\bigskip
Le \emph{Si-sinon}
\begin{Java}
  if ( condition ) {
    instructions
  } else {
    instructions
  }
\end{Java} 
\end{frame}

\begin{frame}[fragile]{Exemple} 
\begin{Java}
import java.util.Scanner;
public class Test {
  public static void main(String[] args) {
      Scanner clavier = new Scanner(System.in);
      int nombre1;

      nombre1 = clavier.nextInt();
      if (nombre1 < 0) {
         System.out.println(nombre1 + " est n�gatif"); 
      }   
  }
}
\end{Java}
\end{frame}

\begin{frame}[fragile]{Exemple} 
\begin{Java}
import java.util.Scanner;
public class Test {
  public static void main(String[] args) {
      Scanner clavier = new Scanner(System.in);
      int nombre1;

      nombre1 = clavier.nextInt();
      System.out.println(nombre1 + " est un nombre ");
      if (nombre1 < 0) {
         System.out.println("n�gatif"); 
      } else {
         System.out.println("positif");   
      }
  }
}
\end{Java}
\end{frame}

\begin{frame}[fragile]{Exercice} 
Comment traduire cet algorithme ?
\begin{Code}
Module test ()
    nombre1: Entier
    Lire nombre1
    Si nombre1 > 0 Alors
        Afficher nombre1, "est positif"
    Sinon
        Si nombre1 = 0 Alors
            Afficher nombre1, "est nul"
        Sinon
            Afficher nombre1, "est n�gatif"
        Fin Si
    Fin Si
Fin Module
\end{Code}
\end{frame}

\begin{frame}[fragile]{Expressions bool�ennes} 
	Les comparateurs
	\begin{center}
		\Large
		\java|<| \; \java|>| \; \java|<=| \; \java|>=| \; \java|==| \; \java|!=|
	\end{center}
	Les op�rateurs bool�ens 
	\begin{center}
		\Large
		\java|&&| (et) \: \java{||} (ou) \: \java|!| (non)
	\end{center}
\end{frame}

\begin{frame}[fragile]{Exemple} 
\begin{Java}
import java.util.Scanner;
public class Exemple {
  public static void main(String[] args) {
      Scanner clavier = new Scanner(System.in);
      int nombre1;

      nombre1 = clavier.nextInt();
      if ((nombre1 % 2) == 0) {
         System.out.println("Le nombre est pair");
      } else {
         System.out.println("Le nombre est impair"); 
      }
  }
}
\end{Java}
\end{frame}

\begin{frame}[fragile]{Exemple} 
\begin{Java}
import java.util.Scanner;
public class Exemple {
  public static void main(String[] args) {
      Scanner clavier = new Scanner(System.in);
      int �ge;

      �ge = clavier.nextInt();
      if ( �ge<21 || �ge>=60 ) {
         System.out.println("Tarif r�duit !");
      }
  }
}
\end{Java}
\end{frame}

\begin{frame}[fragile]{Le \og selon-que\fg} 
Premi�re forme
\begin{Java}
  switch(produit) {
    case "Coca" : 
    case "Sprite" : 
    case "Fanta" :  
        prixDistributeur=60; 
        break;
    case "IceTea" : 
        prixDistributeur=70; 
        break;
    default :       
        prixDistributeur=0; 
        break;
  }
\end{Java}
\begin{itemize}
\item Notez le \java{break}
\item Possible avec : entiers, caract�res et chaines
\end{itemize}
\end{frame}

\begin{frame}[fragile]{Le \og selon-que\fg} 
Deuxi�me forme : la logique suivante
\begin{Code}
  Selon que 
    nb > 0 : Afficher "positif"
    nb = 0 : Afficher "nul"
    autres : Afficher "n�gatif"
  Fin selon que
\end{Code}
s'�crit en Java
\begin{Java}
  if (nb>0) {
      System.out.println("positif");
  } else if (nb==0) {
      System.out.println("nul");
  } else {
      System.out.println("n�gatif");
  }  
\end{Java}
\end{frame}

