% === Cours de Java
% === Fichier principal
\usetheme[secheader]{Madrid}
\usepackage{java}

\title{Le Langage Java}
\subtitle{1\iere\ ann\'ee}
\institute[HEB-�SI]{Haute �cole de Bruxelles --- �cole Sup�rieure d'Informatique}    
\date[2014 --- 2015]{Ann�e acad�mique 2014 / 2015}
\author[]{ \small
  J.~Beleho (bej) \and 
  C.~Leruste (clr) \and
  M.~Codutti (mcd) \and
  P.~Bettens (pbt) \and
  F.~Servais (srv) \and
  C.~Leignel (clg) \and
  D.~Nabet (dna) \and
  J.~Lechien (jlc)
}

\begin{document}

% ===== Page de titre ======
\begin{frame}
\titlepage
\end{frame}

% ===== Table des mati�res ======

\setbeamertemplate{section in toc}[square] 
\setbeamerfont{section in toc}{size=\small}

\begin{frame}{Liste des s�ances }
\vspace{-20pt}
\begin{columns}[T]
  \begin{column}{0.55\textwidth}
	\tableofcontents[hideallsubsections,sections={1}]
	\vspace{-15pt}
	\tableofcontents[hideallsubsections,sections={2}]
	\vspace{-15pt}
	\tableofcontents[hideallsubsections,sections={3}]
	\vspace{-15pt}
	\tableofcontents[hideallsubsections,sections={4}]
	\vspace{-15pt}
	\tableofcontents[hideallsubsections,sections={5}]
	\vspace{-15pt}
	\tableofcontents[hideallsubsections,sections={6}]
  \end{column}
  \begin{column}{0.45\textwidth}
	\tableofcontents[hideallsubsections,sections={7}]
	\vspace{-15pt}
	\tableofcontents[hideallsubsections,sections={8}]
	\vspace{-15pt}
	\tableofcontents[hideallsubsections,sections={9}]
	\vspace{-15pt}
	\tableofcontents[hideallsubsections,sections={10}]
	\vspace{-15pt}
	\tableofcontents[hideallsubsections,sections={11}]
	\vspace{-15pt}
	\tableofcontents[hideallsubsections,sections={12}]
  \end{column}
\end{columns}
\end{frame}


% ===== Les diff�rentess s�ances =====
\include{seance-1}
% ==S�ance 2
\section{D�velopper en \sigle{Java}, premier survol}
\leconwithtoc
% === Cours de Java
% === Chapitre : Introduction

\subsection{La machine virtuelle}

\begin{frame}
	\begin{center}
		Java est \emph{compil�} puis \emph{interpr�t�}.
	\par
	L'interpr�teur Java est la machine virtuelle (\sigle{JVM})
	\par
	Le langage de bas niveau interpr�t� par la \sigle{JVM} est le \emph{\sigle{bytecode}}
\end{center}
\end{frame}

\begin{frame}{La machine virtuelle}
\begin{center}
\includegraphics[scale=0.8]{../img/java-jvm-jvm2} 
\end{center}
\end{frame}

\imgfullh{../img/Sun_and_rain_by_emolawn.jpg}
{\begin{flushright}\large\bf\color{ghostwhite} Avantages et\\inconv�nients\end{flushright}}
	{http://emolawn.deviantart.com/art/Sun-and-rain-85676638}

\note{
	Java serait lent � interpr�ter (langage haut niveau).  
	Donc, introduction d'un niveau interm�diaire, le Bytecode, qui est
 	proche d'un langage d'assemblage, 
 	plus rapide � interpr�ter.
	C'est en fait le langage de la JVM
}


\begin{frame}[fragile]{Exemple: premier programme}
Prenons un exemple \textit{(fichier \code{Hello.java})}
\begin{Java}
// Mon premier programme
public class Hello {
  public static void main(String[] args) {
    System.out.println("Bonjour !");
  }
}
\end{Java}
Compilons-le \texttt{\$ javac Hello.java}
\par
On obtient la version compil�e, le \sigle{bytecode} (\code|Hello.class|)
\par
On peut l'ex�cuter \texttt{\$ java Hello}\\
\texttt{Bonjour !}
\end{frame}

\subsection{Les outils de d�veloppement}

\imgfullh{../img/righttool-3348965007_b3049c78fe_b.jpg}
{\vspace{-1cm}\color{cadetgrey}\Large\bf Fourbir ses armes}
{https://www.flickr.com/photos/ebarney/3348965007/in/photostream/}


\begin{frame}{Les outils de d�veloppement}
Les �ditions de \sigle{Java}
\begin{itemize}
	\item \emph{Java SE} (�dition standard)
	\item \emph{Java ME} (�dition mobile - plus l�ger) 
	\item \emph{Java EE} (�dition entreprise - plus complet)
\end{itemize}
\begin{wrapfigure}{r}{17mm}
	\includegraphics[width=17mm]{../img/java-download.png}
\end{wrapfigure}
\bigskip
O� trouver \code|javac| et \code|java| ?
\par
\emph{\sigle{JRE}} (\textit{Java Runtime Environment}) \\
\emph{\sigle{JDK}} (\textit{Java Development Kit})
\end{frame}


\full[bluepigment]{
	\begin{center}
		\Large\bf\color{azuremist}
		�diter \\ Compiler \\ Ex�cuter
	\end{center}
}

\begin{frame}{Les outils de d�veloppement}
 \par{\bf\Huge 1} 
\begin{itemize}
\item Un \emph{�diteur avec coloration syntaxique}:
	\emph{\sigle{gVim}}, \emph{\sigle{Notepad++}}, \emph{\sigle{nano}}, \dots
\item Gestion manuelle des noms et emplacements des fichiers
\item Compilation et ex�cution en ligne de commande
\end{itemize}
\end{frame}

\begin{frame}{Les outils de d�veloppement}
 \par{\bf\Huge 2}
 \begin{itemize}
	\item Un \textit{\emph{E}nvironnement de \emph{D}�veloppement \emph{I}nt�gr�} : 
	\emph{\sigle{Netbeans}}, \emph{\sigle{Eclipse}}, \dots
  	\item Int�gre tout le processus de d�veloppement  
  \end{itemize}
\end{frame}



% === Cours de Java
% === Chapitre : Survol 

\subsection{Algorithmes s�quentiels (survol)}
\subsection{Structure g�n�rale d'un programme}

\imgfullw{../img/Life_in_a_frame_by_michaelpaleodimos.jpg}
{}
{http://michaelpaleodimos.deviantart.com/art/Life-in-a-frame-39596992}

\begin{frame}[fragile]{Structure g�n�rale du programme}
	\texttt{\small\$cat NomClasse.java}
\begin{Java}
  public class NomClasse {
    // insert code here
  }
\end{Java}
\bigskip
\bigskip
\emph{Attention} \sigle{Java} est sensible � la casse 
\end{frame}

\begin{frame}[fragile]{La m�thode principale}
	\texttt{\small\$cat NomClasse.java}
\begin{Java}
  public class NomClasse {
    public static void main(String[] args) {
      // insert code here     
    }
  }
\end{Java}
\end{frame}


\begin{frame}[fragile]{Les variables}
Les types disponibles
\begin{center}
\begin{tabular}{r|l}
En Logique & En  Java \\ \hline
Entier & \java|int| \\ 
R�el & \java|double| \\ 
Chaine & \java|String| \\ 
Caract�re & \java|char| \\ 
Bool�en & \java|boolean| \\  
\end{tabular} 
\end{center}
Exemple de d�claration
\begin{Java}
   int nb1;
\end{Java}
\end{frame}

\begin{frame}[fragile]{L'assignation et les calculs}
	L'assignation se fait via le symbole 
	\begin{center}
		\huge\java|=|
	\end{center}
\begin{Java}
   nb1 = 1;
\end{Java}
\bigskip
Op�rateurs :
\begin{center}
	\Large
    \java|+| \java|-| \java|*| \java|/| \java|%|  
\end{center}
\end{frame}

\begin{frame}[fragile]{Exemple}
	\texttt{\small\$cat Moyenne.java}
\begin{Java}
  public class Moyenne {
    public static void main(String[] args) {

      double nombre1;
      double nombre2;
      double moyenne;

      nombre1 = 34345;
      nombre2 = -3213213;
      moyenne = (nombre1 + nombre2) / 2;    
      System.out.println(moyenne); 
    }
  }
\end{Java}
\end{frame}

\begin{frame}[fragile]{Exemple}
	\texttt{\small\$cat Moyenne.java}
\begin{Java}
  public class Moyenne {
    public static void main(String[] args) {

      int nombre1 = 34345;
      int nombre2 = -321321;
      double moyenne;

      // division r�elle car un des 2 op�randes est r�el
      moyenne = (nombre1 + nombre2) / 2.0;    
      System.out.println("La moyenne est " + moyenne); 
    }
  }
\end{Java}
\end{frame}


\begin{frame}[fragile]{Lire au clavier}
	\par\textit{Les applications modernes pr�f�rent les lectures dans des champs de saisies.}
	\par Dans une console, voici une mani�re de faire
	\medskip
	\par\emph{Exemple}
  \begin{Java}
  import java.util.Scanner;
  // ...
  Scanner clavier = new Scanner(System.in);
  // ...
  nombre1 = clavier.nextInt();
  \end{Java}
\end{frame}

\begin{frame}[fragile]{Lire au clavier - Exemple}
	\texttt{\small\$cat Test.java}
\begin{Java}
import java.util.Scanner;

public class Test {
  public static void main(String[] args) {
      Scanner clavier = new Scanner(System.in);
      double nombre1;
      double nombre2;
      double moyenne;

      nombre1 = clavier.nextDouble();
      nombre2 = clavier.nextDouble();
      moyenne = (nombre1 + nombre2) / 2.0;
      System.out.println(moyenne);    
  }
}
\end{Java}
\end{frame}

\begin{frame}{Lire au clavier}
\begin{center}
\begin{tabular}{r|l}
Pour lire\dots & on �crit\dots \\ \hline
un entier & \java|nextInt()| \\ 
un r�el & \java|nextDouble()| \\ 
un bool�en & \java|nextBoolean()| \\ 
un mot & \java|next()|\\  
une ligne & \java|nextLine()|\\  
un caract�re & \java|next().charAt(0)| \\ 
\end{tabular} 
\end{center}
\end{frame}

\subsection{Constantes}

\begin{frame}[fragile]{Constante locale}
	Une \textbf{constante} s'�crit gr�ce � \java|final| 
\par\medskip
\par\emph{Exemple}
\begin{Java}
  final int X = 1;
  final int Y;
  Y = 2*X;
  X = 2; // Erreur : poss�de d�j� une valeur
  Y = 3; // Idem
\end{Java}
\end{frame}

\subsection{Conventions}

\imgfullh{../img/people_think_theyre_different_by_nanfe-d3bzl5i_mirror.jpg}
{\vspace{19mm}\LARGE\bf\color{brickred} Conventions d'�criture}
{http://nanfe.deviantart.com/art/people-think-theyre-different-201534678}

\subsection{Conventions de noms}

\begin{frame}[fragile]{Conventions de noms}
Pour une variable :
\begin{itemize}
  \item Tout mettre en minuscules
  \item Sauf les d�buts de noms compos�s en majuscule
\end{itemize}

Pour une constante :
\begin{itemize}
  \item Tout mettre en majuscules
  \item Utiliser \_ pour s�parer les mots
\end{itemize}

Dans tous les cas : �tre explicite
\end{frame}

\subsection{Commentaires}

\begin{frame}[fragile]{Le commentaire}
Plusieurs mani�res d'ajouter un commentaire 
  \begin{Java}
  // Commentaire sur une ligne
  /* Commentaire sur
     plusieurs lignes */
  \end{Java}
\end{frame}




% ==S�ance 3
\section{Survol s�quentiel}
\leconwithtoc
% === Cours de Java
% === Chapitre : Survol 

\subsection{Algorithmes s�quentiels (survol)}
\subsection{Structure g�n�rale d'un programme}

\imgfullw{../img/Life_in_a_frame_by_michaelpaleodimos.jpg}
{}
{http://michaelpaleodimos.deviantart.com/art/Life-in-a-frame-39596992}

\begin{frame}[fragile]{Structure g�n�rale du programme}
	\texttt{\small\$cat NomClasse.java}
\begin{Java}
  public class NomClasse {
    // insert code here
  }
\end{Java}
\bigskip
\bigskip
\emph{Attention} \sigle{Java} est sensible � la casse 
\end{frame}

\begin{frame}[fragile]{La m�thode principale}
	\texttt{\small\$cat NomClasse.java}
\begin{Java}
  public class NomClasse {
    public static void main(String[] args) {
      // insert code here     
    }
  }
\end{Java}
\end{frame}


\begin{frame}[fragile]{Les variables}
Les types disponibles
\begin{center}
\begin{tabular}{r|l}
En Logique & En  Java \\ \hline
Entier & \java|int| \\ 
R�el & \java|double| \\ 
Chaine & \java|String| \\ 
Caract�re & \java|char| \\ 
Bool�en & \java|boolean| \\  
\end{tabular} 
\end{center}
Exemple de d�claration
\begin{Java}
   int nb1;
\end{Java}
\end{frame}

\begin{frame}[fragile]{L'assignation et les calculs}
	L'assignation se fait via le symbole 
	\begin{center}
		\huge\java|=|
	\end{center}
\begin{Java}
   nb1 = 1;
\end{Java}
\bigskip
Op�rateurs :
\begin{center}
	\Large
    \java|+| \java|-| \java|*| \java|/| \java|%|  
\end{center}
\end{frame}

\begin{frame}[fragile]{Exemple}
	\texttt{\small\$cat Moyenne.java}
\begin{Java}
  public class Moyenne {
    public static void main(String[] args) {

      double nombre1;
      double nombre2;
      double moyenne;

      nombre1 = 34345;
      nombre2 = -3213213;
      moyenne = (nombre1 + nombre2) / 2;    
      System.out.println(moyenne); 
    }
  }
\end{Java}
\end{frame}

\begin{frame}[fragile]{Exemple}
	\texttt{\small\$cat Moyenne.java}
\begin{Java}
  public class Moyenne {
    public static void main(String[] args) {

      int nombre1 = 34345;
      int nombre2 = -321321;
      double moyenne;

      // division r�elle car un des 2 op�randes est r�el
      moyenne = (nombre1 + nombre2) / 2.0;    
      System.out.println("La moyenne est " + moyenne); 
    }
  }
\end{Java}
\end{frame}


\begin{frame}[fragile]{Lire au clavier}
	\par\textit{Les applications modernes pr�f�rent les lectures dans des champs de saisies.}
	\par Dans une console, voici une mani�re de faire
	\medskip
	\par\emph{Exemple}
  \begin{Java}
  import java.util.Scanner;
  // ...
  Scanner clavier = new Scanner(System.in);
  // ...
  nombre1 = clavier.nextInt();
  \end{Java}
\end{frame}

\begin{frame}[fragile]{Lire au clavier - Exemple}
	\texttt{\small\$cat Test.java}
\begin{Java}
import java.util.Scanner;

public class Test {
  public static void main(String[] args) {
      Scanner clavier = new Scanner(System.in);
      double nombre1;
      double nombre2;
      double moyenne;

      nombre1 = clavier.nextDouble();
      nombre2 = clavier.nextDouble();
      moyenne = (nombre1 + nombre2) / 2.0;
      System.out.println(moyenne);    
  }
}
\end{Java}
\end{frame}

\begin{frame}{Lire au clavier}
\begin{center}
\begin{tabular}{r|l}
Pour lire\dots & on �crit\dots \\ \hline
un entier & \java|nextInt()| \\ 
un r�el & \java|nextDouble()| \\ 
un bool�en & \java|nextBoolean()| \\ 
un mot & \java|next()|\\  
une ligne & \java|nextLine()|\\  
un caract�re & \java|next().charAt(0)| \\ 
\end{tabular} 
\end{center}
\end{frame}

\subsection{Constantes}

\begin{frame}[fragile]{Constante locale}
	Une \textbf{constante} s'�crit gr�ce � \java|final| 
\par\medskip
\par\emph{Exemple}
\begin{Java}
  final int X = 1;
  final int Y;
  Y = 2*X;
  X = 2; // Erreur : poss�de d�j� une valeur
  Y = 3; // Idem
\end{Java}
\end{frame}

\subsection{Conventions}

\imgfullh{../img/people_think_theyre_different_by_nanfe-d3bzl5i_mirror.jpg}
{\vspace{19mm}\LARGE\bf\color{brickred} Conventions d'�criture}
{http://nanfe.deviantart.com/art/people-think-theyre-different-201534678}

\subsection{Conventions de noms}

\begin{frame}[fragile]{Conventions de noms}
Pour une variable :
\begin{itemize}
  \item Tout mettre en minuscules
  \item Sauf les d�buts de noms compos�s en majuscule
\end{itemize}

Pour une constante :
\begin{itemize}
  \item Tout mettre en majuscules
  \item Utiliser \_ pour s�parer les mots
\end{itemize}

Dans tous les cas : �tre explicite
\end{frame}

\subsection{Commentaires}

\begin{frame}[fragile]{Le commentaire}
Plusieurs mani�res d'ajouter un commentaire 
  \begin{Java}
  // Commentaire sur une ligne
  /* Commentaire sur
     plusieurs lignes */
  \end{Java}
\end{frame}



%% === Cours de Java
% === Chapitre : Survol 

\subsection{Alternatives (survol)}

\imgfullh{../img/The_choice_by_vallo29.jpg}
{\centering\color{bluepigment}\Large\bf Alternatives}
{http://vallo29.deviantart.com/art/The-choice-150871274}

\begin{frame}[fragile]{Instructions de choix}
Le \emph{Si}
\begin{Java}
  if ( condition ) {
    instructions
  }
\end{Java}
\bigskip
Le \emph{Si-sinon}
\begin{Java}
  if ( condition ) {
    instructions
  } else {
    instructions
  }
\end{Java} 
\end{frame}

\begin{frame}[fragile]{Exemple} 
\begin{Java}
import java.util.Scanner;
public class Test {
  public static void main(String[] args) {
      Scanner clavier = new Scanner(System.in);
      int nombre1;

      nombre1 = clavier.nextInt();
      if (nombre1 < 0) {
         System.out.println(nombre1 + " est n�gatif"); 
      }   
  }
}
\end{Java}
\end{frame}

\begin{frame}[fragile]{Exemple} 
\begin{Java}
import java.util.Scanner;
public class Test {
  public static void main(String[] args) {
      Scanner clavier = new Scanner(System.in);
      int nombre1;

      nombre1 = clavier.nextInt();
      System.out.println(nombre1 + " est un nombre ");
      if (nombre1 < 0) {
         System.out.println("n�gatif"); 
      } else {
         System.out.println("positif");   
      }
  }
}
\end{Java}
\end{frame}

\begin{frame}[fragile]{Exercice} 
Comment traduire cet algorithme ?
\begin{Code}
MODULE test ()
    nombre1: Entier
    LIRE nombre1
    SI nombre1 > 0 ALORS
        ECRIRE nombre1, "est positif"
    SINON
        SI nombre1 = 0 ALORS
            ECRIRE nombre1, "est nul"
        SINON
            ECRIRE nombre1, "est n�gatif"
        FIN SI
    FIN SI
FIN MODULE
\end{Code}
\end{frame}

\begin{frame}[fragile]{Expressions bool�ennes} 
	Les comparateurs
	\begin{center}
		\Large
		\java|<| \; \java|>| \; \java|<=| \; \java|>=| \; \java|==| \; \java|!=|
	\end{center}
	Les op�rateurs bool�ens 
	\begin{center}
		\Large
		\java|&&| (et) \: \java{||} (ou) \: \java|!| (non)
	\end{center}
\end{frame}

\begin{frame}[fragile]{Exemple} 
\begin{Java}
import java.util.Scanner;
public class Exemple {
  public static void main(String[] args) {
      Scanner clavier = new Scanner(System.in);
      int nombre1;

      nombre1 = clavier.nextInt();
      if ((nombre1 % 2) == 0) {
         System.out.println("Le nombre est pair");
      } else {
         System.out.println("Le nombre est impair"); 
      }
  }
}
\end{Java}
\end{frame}

\begin{frame}[fragile]{Exemple} 
\begin{Java}
import java.util.Scanner;
public class Exemple {
  public static void main(String[] args) {
      Scanner clavier = new Scanner(System.in);
      int �ge;

      �ge = clavier.nextInt();
      if ( �ge<21 || �ge>=60 ) {
         System.out.println("Tarif r�duit !");
      }
  }
}
\end{Java}
\end{frame}

\begin{frame}[fragile]{Le \og selon-que\fg} 
Premi�re forme
\begin{Java}
  switch(produit) {
    case "Coca" : 
    case "Sprite" : 
    case "Fanta" :  
        prixDistributeur=60; 
        break;
    case "IceTea" : 
        prixDistributeur=70; 
        break;
    default :       
        prixDistributeur=0; 
        break;
  }
\end{Java}
\begin{itemize}
\item Notez le \java{break}
\item Possible avec : entiers, caract�res et chaines
\end{itemize}
\end{frame}

\begin{frame}[fragile]{Le \og selon-que\fg} 
Deuxi�me forme : la logique suivante
\begin{Code}
  Selon que 
    nb > 0 : �crire "positif"
    nb = 0 : �crire "nul"
    autres : �crire "n�gatif"
  Fin selon que
\end{Code}
s'�crit en Java
\begin{Java}
  if (nb>0) {
      System.out.println("positif");
  } else if (nb==0) {
      System.out.println("nul");
  } else {
      System.out.println("n�gatif");
  }  
\end{Java}
\end{frame}



% ==S�ance 4
\section{Lisibilit� et notions de modules}
\leconwithtocquote
{\og Any fool can write code that a computer can understand.
\\Good programmers write code that humans can understand. \fg
\\Martin Fowler}
% === Cours de Java

\subsection{�crire du code lisible}

\full[bluepigment]{
	\texttt{\color{white}
		public class h\{public static void main(String[] args)\{System.out.println("Hi");\}\}
	}
}

\begin{frame}{Lisibilit� du code}
Un code est \emph{souvent lu};
  \begin{itemize}
  \item lorsqu'il est �crit / mis au point;
  \item correction de bug; 
  \item �volution du code;
  \end{itemize}
\bigskip
$\Longrightarrow$ \emph{La lisibilit� est essentielle}
\end{frame}

\begin{frame}[fragile]{Lisibilit� du code: indentation}
	\par{\bf\Huge 1}
	\par \emph{Indenter} correctement son code 
\end{frame}

\begin{frame}[fragile]{Lisibilit� du code: choix des noms}
	\par{\bf\Huge 2}
	\par Bien choisir le \emph{nom des variables}
	\bigskip
	\begin{Java}
  int u=clavier.nextInt(),n=clavier.nextInt(),
  t=clavier.nextInt();
  double p=u*n*(1+t/100.0);
  System.out.println(p);
	\end{Java}
\end{frame}

\imgfullh{../img/screenshot-calculprixvente.png}{}{http://pit.namok.be} % ;-)

\begin{frame}[fragile]{Lisibilit� du code: d�composition}
	\par{\bf\Huge 3}
	\par \emph{D�composer} les expressions trop longues
	\bigskip
	\bigskip
	\begin{Java}
  �Payer = prixUnitaireHTVA * (1 + tauxTVA/100.0) * nombreArticles;

  prixUnitaireTTC = prixUnitaireHTVA * (1 + tauxTVA/100.0);                               
  �Payer = prixUnitaireTTC * nombreArticles;
	\end{Java}
\end{frame}

\begin{frame}[fragile]{Lisibilit� du code: constantes}
	\par{\bf\Huge 4}
	\par Utiliser des \emph{constantes}
	\vspace{1.5cm}
\begin{Java}                                                 
  final double TAUX_TVA = 0.21;
\end{Java}
\end{frame}

\subsection{�crire du code illisible}

\begin{frame}[fragile]{Illisibilit� du code}
	\par{\bf\Huge\color{brickred} -1} 
	\par Surcharger de \emph{commentaires}
	\vspace{1.5cm}
	\par\textit{Un commentaire explique ce que le code fait mais pas comment il le fait}
\end{frame}

% TODO ajouter une image avec interdit 
% image avec comme texte C'est tout
% supprimer slide suivant

\imgfullh{../img/rule.jpg}
	{\vspace{4cm}\LARGE\bf\color{ghostwhite}D'autres conventions ?\vspace{-4cm}}
	{https://www.flickr.com/photos/fuzzhead/7310236/sizes/o/}

\begin{frame}[fragile]{Conventions d'�criture}
	Conventions d'�criture \sigle{Java}
\begin{itemize}
	\item {\footnotesize\href{http://www.oracle.com/technetwork/java/javase/documentation/codeconvtoc-136057.html}
		{http://www.oracle.com/...codeconv...}}
\end{itemize} 
\end{frame}

\subsection{Refactorisation}

\begin{frame}{La refactorisation}
	\par{\bf Refactorisation}
	\medskip
	\par\centering\textit{Improving the design of existing code}
	\bigskip
	\begin{flushright}
		\color{cadetgrey}\sigle{JUnit}\: VCS (git/svn)
	\end{flushright}
\end{frame}



% === Cours de Java
% === Chapitre : Survol 

\subsection{Code modulaire (survol)}

\imgfullh{../img/cut_and_fold_by_alltelleringet-d5jmcpu.jpg}
	{
		\begin{flushright}
			\vspace{3.5cm}
			\Large\bf\color{honeydew} D�couper le code
			\vspace{-3.5cm}
		\end{flushright}
		}
	{http://alltelleringet.deviantart.com/art/Cut-and-Fold-335286498}

\begin{frame}{D�couper du code}
\emph{Pourquoi ?}
\begin{itemize}
\item Pour le r�utiliser 
\item Pour scinder la difficult�
\item Pour faciliter le d�verminage
\item Pour accroitre la lisibilit�
\item Pour diviser le travail
\end{itemize} 
\end{frame}

\begin{frame}{D�couper du code}
\emph{Comment ?}
\begin{itemize}
\item $\exists$ un \emph{nom qui d�crit tout ce qu'il fait}
\item Il r�sout un \emph{sous-probl�me} bien pr�cis
\item Il est \emph{fortement document�}
\item Il est le plus \emph{g�n�ral possible}
\item Il tient sur une page
\end{itemize} 
\medskip
En \sigle{Java} on dit \emph{m�thode} et pas \emph{module}
\end{frame}

\full[bluepigment]{
	\begin{center}
		\Large\color{white}\textbf{Appel} de m�thode
	\end{center}
}

\begin{frame}{Appel d'une m�thode}
Une m�thode est une \emph{boite noire}
\begin{center}
\includegraphics[scale=.6]{../img/methode}
\end{center}
Pour l'utiliser, il faut connaitre\,:
\begin{itemize}
\item son nom;
\item ce dont elle a besoin;
\item ce qu'elle retourne;
\item mais \emph{pas comment} elle fait;
\end{itemize}
\end{frame}

\imgfullh{../img/darkvador-magician.jpg}
{
		\begin{center}
			\color{white}\large
			Pas de \textbf{comment} ? \\
			Un simple \textbf{quoi}
		\end{center}
}
{https://www.flickr.com/photos/st3f4n/14480265204/sizes/l}

\begin{frame}[fragile]{Appel d'une m�thode}
� partir du code d'une \emph{autre classe} 
  \begin{itemize}
  \item \java|NomClasse.nomM�thode(...)|
  \item \emph{Exemples} :
    \begin{Java}
double racine = Math.sqrt(4.0);
double al�atoire = Math.random();
int nb = -10;
int absolu = Math.abs(nb);
    \end{Java}
  \end{itemize}
  \bigskip
  \textit{Un appel de m�thode au sein de la classe ne n�cessite pas de r�p�ter
  le nom de la classe}
\end{frame}

\full[bluepigment]{
	\begin{center}
		\Large\color{azuremist}\textbf{D�finition} d'une m�thode
	\end{center}
}

\begin{frame}[fragile]{D�finition d'une m�thode}
	\textbf{Syntaxe}
	\bigskip
\begin{Java}
	public static <typeRetour> nomM�thode ([param�tre, param�tre, ...]) {  
    // instructions
    <return r�sultat>;  
}
\end{Java}
\end{frame}

\begin{frame}[fragile]{D�finition d'une m�thode}
\emph{Exemple} : la moyenne de 2 r�els
\begin{Java}
public static double moyenne ( double nb1, double nb2 ) {  
    double moyenne = (nb1 + nb2) / 2.0;
    return moyenne;
}
\end{Java}
\begin{itemize}
\item Appel possible (si dans la m�me classe)
\begin{Java}
  double cote = moyenne(12.5, 17.5);
\end{Java}
\end{itemize}
\end{frame}

\begin{frame}[fragile]{D�finition d'une m�thode}
\emph{Exemple} : la valeur absolue
\begin{Java}
public static int absolu ( int nb ) {  
    int abs;
    if (nb<0) {
        abs = -nb;
    } else {
        abs = nb;
    }
    return abs;
}
\end{Java}
\begin{itemize}
\item Exemples d'appels
\begin{Java}
   int r�sultat = absolu(4);
   int �cart = -10;
   int �cartAbsolu = absolu(�cart);
\end{Java}
\end{itemize}
\end{frame}

\begin{frame}[fragile]{D�finition d'une m�thode}
\emph{Exemple} :
\begin{Java}
public static void pr�senter (String nomPgm) {
   System.out.println("Programme "+nomPgm);
} 
\end{Java}
\begin{itemize}
\item Exemple d'appel
\begin{Java}
   pr�senter("moyenne de 2 nombres");
\end{Java}
\end{itemize}
\end{frame}

\begin{frame}[fragile]{D�finition d'une m�thode}
\emph{Exemple} :
\begin{Java}
public static int lireEntier () {
    Scanner clavier = new Scanner(System.in);
    int nb;
    System.out.println("Entrez un nombre entier!");
    nb = clavier.nextInt();
    return nb;
}
\end{Java}
\begin{itemize}
\item Exemple d'appel
\begin{Java}
   int nb = lireEntier();
\end{Java}
\end{itemize}
\end{frame}

\begin{frame}[fragile]{Commentaire d'une m�thode}
	Documentation de \java{Math.abs}
	\begin{flushright}
	\includegraphics[width=0.91\textwidth]{../img/screenshot-jdk8-javadoc-abs.png}
	\end{flushright}
\end{frame}

\begin{frame}[fragile]{Commentaire d'une m�thode}
Il est essentiel de commenter chaque m�thode.
\medskip
\emph{Exemple} : la valeur absolue
\begin{Java}
/**
 * Calcul de la valeur absolue.
 * @param nb le nombre dont on veut la valeur absolue.
 * @return la valeur absolue de <code>nb</code>
 */
public static int absolu ( int nb ) {  
...
}
\end{Java}
\end{frame}


\begin{frame}[fragile]{Un exemple complet}
\begin{Java}[basicstyle=\scriptsize]
import java.util.Scanner;

public class MaxEntiers { 

  /**
   * Donne le maximum de 2 nombres.
   * @param nb1 le premier nombre.
   * @param nb2 le deuxi�me nombre.
   * @return la valeur la plus grande entre <code>nb1</code> et <code>nb2</code>
   */
  public static int max ( int nb1, int nb2 ) { 
       int max=0; 
       if (nb1 > nb2) {
           max = nb1;
       } else {
           max = nb2;
       }
       return max;
  }
\end{Java}
\begin{flushright}
{\tiny (\dots)}
\end{flushright}
\end{frame}

\begin{frame}[fragile]{Un exemple complet}
\begin{Java}[basicstyle=\scriptsize]
  /**
   * Lit un nombre entier. 
   * Le nombre est lu sur l'entr�e standard (le clavier).
   * @return le nombre entier lu.
   */
  public static int lireEntier () {
      Scanner clavier = new Scanner(System.in);
      System.out.println("Entrez un nombre entier!");
      return clavier.nextInt();
  }
 
  /**
   * Affiche le maximum de 2 nombres entr�s au clavier.
   * @param args pas utilis�.
   */
  public static void main ( String[] args ) { 
      int max;  // Le max des nombres lus
      int nb1, nb2; // Chacun des nombres lus
      nb1 = lireEntier();
      nb2 = lireEntier();
      max = max(nb1,nb2);
      System.out.println("max = " + max);
  }
}
\end{Java}
\end{frame}

\imgfullh{../img/Containers_by_g4mbit.jpg}
{
	\vspace{4cm}
	\Large\bf\color{white} Passage de param�tres
	\vspace{-4cm}
}
{http://g4mbit.deviantart.com/art/Containers-139196962}

\begin{frame}{Passage de param�tres}
En algorithmique 3 passages de param�tres :
  \begin{itemize}
  \item en entr�e, en sortie, en entr�e-sortie
  \end{itemize} 
\par\bigskip
En Java, uniquement \emph{par valeur}
  \begin{itemize}
  \item = la valeur est copi�e dans le param�tre
  \item $\simeq$ param�tre en entr�e
  \end{itemize} 
\end{frame}



% ==S�ance 5
\section{Notion de package et survol des structures r�p�titives}
\leconwithtoc


% === Cours de Java

\subsection{Organiser le code}

\imgfullw{../img/fruitetlegumes.jpg}
	{
		\begin{flushright}
		\vspace{1cm}
		\LARGE\bf\color{white}
		Rangement
		\vspace{-1cm}
	\end{flushright}
	}
	{https://www.flickr.com/photos/wien/418840561}

\begin{frame}{Le groupement en package}
	Toutes les classes de l'\emph{API} Java sont regroup�es logiquement \dots 
	en \emph{package}
	\bigskip
\begin{center}
  \includegraphics[scale=.6]{../img/package}
\end{center}
\end{frame}

\begin{frame}{La notion de package}
Un \emph{package} donne un nom complet � une classe
\begin{itemize}
    \item {\java|mon.paquet.MaClasse|},
    \item {\java|be.heb.esi.java1.MaClasse|}, 
    \item {\java|java.util.Scanner|},
	\item {\java|org.apache.struts2.components.Anchor|}
\end{itemize} 
\note[item]{regroupement de classes li�es}
\note[item]{unicit� des noms de classe}
\note[item]{identifieurs s�par�s par des .}
\note[item]{tout en minuscules}
\note[item]{adresse internet invers�e (unicit�)}
\end{frame}

\begin{frame}[fragile]{Utilisation}
Pour utiliser une classe 
\begin{itemize}
\item mettre le nom \emph{qualifi�} (complet)
\begin{Java}
  java.util.Calendar now = java.util.Calendar.getInstance();
\end{Java}
\item ou utiliser \java{import} qui cr�e un raccourci
\begin{Java}[basicstyle=\scriptsize]
  import java.util.Calendar;
  public Test {
  ...
    Calendar now = Calendar.getInstance();
  ...
  }
\end{Java}
\end{itemize}
\textbf{Cas particulier} Le package \java|java.lang| est import� implicitement
\end{frame}

\begin{frame}{Utilisation}
Comment savoir comment utiliser les classes et m�thodes ? 
\bigskip
\begin{flushright}
	En lisant la \emph{javadoc}
\end{flushright}
\end{frame}

\imgfullh{../img/api}{}
{(http://download.oracle.com/javase/7/docs/api/)}

\begin{frame}{Utilisation}
On peut y lire le nom du package 
\begin{center}
\includegraphics[scale=.6]{../img/api-package}
\end{center}
et la description de la m�thode
\begin{center}
\includegraphics[scale=.6]{../img/api-methode}
\\{\small \textit{On verra comment produire une \emph{javadoc} similaire pour son code}}
\end{center}
\end{frame}

\full[bluepigment]{
	\begin{center}
		\Large\bf\color{azuremist}
		Comment cr�er mes propres \textit{packages} ?
	\end{center}
}

\begin{frame}[fragile]{Cr�er ses packages}
	Commande: \java|package <nom du paquet>|
	\bigskip
\begin{Java}
package be.heb.esi.java1;
public class Test { 
  // Nom complet : be.heb.esi.java1.Test
}
\end{Java}
\end{frame}

\begin{frame}[fragile]{Cr�er ses packages}
Qu'est-ce qui va changer en pratique ?
\begin{itemize}
\item La compilation ne change pas : 
 \begin{Java}
 javac NomClasse.java
 \end{Java}
\item L'ex�cution change : 
 \begin{Java}
 java nom.paquet.NomClasse
 \end{Java}
\end{itemize}
\bigskip
Cela a une incidence sur l'endroit o� placer le \emph{bytecode}
\end{frame}

\imgfullh{../img/api}
{	
	\color{bluepigment}\Large\bf
	\colorbox{aliceblue}{%
		\parbox{.95\linewidth}{%
		\centering Connaitre l'API,\\ c'est gagner du temps ! }
	}
}
{(http://download.oracle.com/javase/7/docs/api/)}

\note[item]{En profiter pour insister sur la diff�rence entre des appels 
de m�thodes dans la m�me classe, dans des classes diff�rentes de m�me package, 
de packages diff�rents.}

%%%
\subsection{�tude de cas}
\full[bluepigment]{
	\begin{center}
		\Large\bf\color{azuremist}
		\href{http://thenounproject.com/term/consultation/39938/}{%
			\includegraphics[width=50px]{../img/icon_39938_transwhite.png}}
		\par\bigskip
		�tude de cas
	\end{center}
}


% === Cours de Java
% === Chapitre : Survol des instructions r�p�titives

\subsection{Les boucles (survol)}

\imgfullw{../img/looptheloop}
	{
		\begin{flushright}
			\color{aliceblue}\Large\it
			\vspace{-3cm}
			Loop the loop !\hspace{1cm}
		\end{flushright}
	}
	{https://www.flickr.com/photos/sirispjelkavik/2801926735/sizes/o/}

\begin{frame}[fragile]{Instructions r�p�titives}
Le \emph{Tant que} : 
\begin{Java}
  while ( condition ) {
    instructions
  }
\end{Java} 
\bigskip
\emph{Exemple} :
\begin{Java}
  int puissance = 1;
  while ( puissance < 1000 ) {
    System.out.println(puissance);
    puissance = 2 * puissance;
  }
\end{Java}
\end{frame}

\begin{frame}[fragile]{Exemple} 
\begin{Java}
import java.util.Scanner;
public class Exemple {
  /**
   * Affiche la somme d'entiers positifs entr�s au clavier.
   * S'arr�te d�s qu'une valeur nulle ou n�gative est donn�e.
   * @param args non utilis�
   */
  public static void main(String[] args) {
      Scanner clavier = new Scanner(System.in);
      int nb;
      int somme = 0;
      nb = clavier.nextInt();
      while ( nb > 0 ) {
         somme = somme + nb;
         nb = clavier.nextInt();
      }
      System.out.println(somme);
  }
}
\end{Java}
\end{frame}

\begin{frame}[fragile]{Instructions r�p�titives}
Le \emph{Pour} : 
\begin{Java}
  for ( int i=d�but; i<=fin; i=i+pas ) {
    instructions
  }
\end{Java} 
\bigskip
\emph{Exemple} :
\begin{Java}
  for ( int i=1; i<=10; i=i+1 ) {
    System.out.println(i);
  }
\end{Java}
\end{frame}

\begin{frame}[fragile]{Exemple} 
\begin{Java}
public class Exemple {
  /**
   * Affiche la somme des nombres pairs entre 2 et 100.
   * @param args non utilis�
   */
  public static void main(String[] args) {
      int somme;

      somme = 0;
      for ( int i=2; i<=100; i=i+2 ) {
         somme = somme + i;
      }
      System.out.println(somme);
  }
}
\end{Java}
\end{frame}

\begin{frame}[fragile]{Exemple} 
\begin{Java}
public class Exemple {
  /**
   * Affiche un compte � rebours � partir de 10.
   * @param args non utilis�
   */
  public static void main(String[] args) {

      for ( int i=10; i>=1; i=i-1 ) {
         System.out.println(i);
      }
      System.out.println("Partez !");
  }
}
\end{Java}
\end{frame}

\begin{frame}[fragile]{Instructions r�p�titives}
\java|i++| est un raccourci pour \java|i=i+1|
\\\bigskip
\emph{Exemple} :
\begin{Java}
  for ( int i=1; i<=n; i++ ) {
    System.out.println(i);
  }
\end{Java} 
\end{frame}

\full[bluepigment]{
	\begin{center}
		\Large\bf\color{azuremist}
		�tude de cas\\
		\it
		Lecture d'une donn�e enti�re positive
	\end{center}
}

\begin{frame}[fragile]{�tude de cas} 
\begin{itemize}
\item \emph{�tape 1} : lire un entier
\end{itemize}
\begin{Java}
/**
  * Lit un entier au clavier.
  * Les valeurs non enti�res sont pass�es.
  * @return l'entier lu.
  */
public static int lireEntier() {
    Scanner clavier = new Scanner(System.in);
    int nb;
    // Tant que ce n'est pas un entier au clavier
    while ( !clavier.hasNextInt() ) {
        clavier.next(); // le lire, le passer
    }
    nb = clavier.nextInt();
    return nb;
}
\end{Java}
\end{frame}

\begin{frame}[fragile]{�tude de cas} 
\begin{itemize}
\item \emph{�tape 2} : lire un entier positif
\end{itemize}
\begin{Java}
/**
  * Lit un entier au clavier.
  * Les valeurs non enti�res, nulles ou n�gatives sont pass�es.
  * @return l'entier lu.
  */
public static int lirePositif() {
    int nb;
    nb = lireEntier();
    while (nb<=0) {
      nb = lireEntier();
    }
    return nb;
}
\end{Java}
\end{frame}

\begin{frame}[fragile]{�tude de cas}
	\begin{itemize}
		\item \emph{�tape 3}: un exemple de main
	\end{itemize}
	\begin{Java}
/**
 * Un exemple de main.
 */
 public static void main(String[] args){
	int nombreLu;

	System.out.print("Entre un entier positif: ");
	nombreLu = lirePositif();
 }
	\end{Java}
\end{frame}






% ==S�ance 6
\section{Les alternatives et la gestion des erreurs}
\leconwithtoc

% === Cours de Java
% === Chapitre : Survol 

\subsection{Alternatives (survol)}

\imgfullh{../img/The_choice_by_vallo29.jpg}
{\centering\color{bluepigment}\Large\bf Alternatives}
{http://vallo29.deviantart.com/art/The-choice-150871274}

\begin{frame}[fragile]{Instructions de choix}
Le \emph{Si}
\begin{Java}
  if ( condition ) {
    instructions
  }
\end{Java}
\bigskip
Le \emph{Si-sinon}
\begin{Java}
  if ( condition ) {
    instructions
  } else {
    instructions
  }
\end{Java} 
\end{frame}

\begin{frame}[fragile]{Exemple} 
\begin{Java}
import java.util.Scanner;
public class Test {
  public static void main(String[] args) {
      Scanner clavier = new Scanner(System.in);
      int nombre1;

      nombre1 = clavier.nextInt();
      if (nombre1 < 0) {
         System.out.println(nombre1 + " est n�gatif"); 
      }   
  }
}
\end{Java}
\end{frame}

\begin{frame}[fragile]{Exemple} 
\begin{Java}
import java.util.Scanner;
public class Test {
  public static void main(String[] args) {
      Scanner clavier = new Scanner(System.in);
      int nombre1;

      nombre1 = clavier.nextInt();
      System.out.println(nombre1 + " est un nombre ");
      if (nombre1 < 0) {
         System.out.println("n�gatif"); 
      } else {
         System.out.println("positif");   
      }
  }
}
\end{Java}
\end{frame}

\begin{frame}[fragile]{Exercice} 
Comment traduire cet algorithme ?
\begin{Code}
MODULE test ()
    nombre1: Entier
    LIRE nombre1
    SI nombre1 > 0 ALORS
        ECRIRE nombre1, "est positif"
    SINON
        SI nombre1 = 0 ALORS
            ECRIRE nombre1, "est nul"
        SINON
            ECRIRE nombre1, "est n�gatif"
        FIN SI
    FIN SI
FIN MODULE
\end{Code}
\end{frame}

\begin{frame}[fragile]{Expressions bool�ennes} 
	Les comparateurs
	\begin{center}
		\Large
		\java|<| \; \java|>| \; \java|<=| \; \java|>=| \; \java|==| \; \java|!=|
	\end{center}
	Les op�rateurs bool�ens 
	\begin{center}
		\Large
		\java|&&| (et) \: \java{||} (ou) \: \java|!| (non)
	\end{center}
\end{frame}

\begin{frame}[fragile]{Exemple} 
\begin{Java}
import java.util.Scanner;
public class Exemple {
  public static void main(String[] args) {
      Scanner clavier = new Scanner(System.in);
      int nombre1;

      nombre1 = clavier.nextInt();
      if ((nombre1 % 2) == 0) {
         System.out.println("Le nombre est pair");
      } else {
         System.out.println("Le nombre est impair"); 
      }
  }
}
\end{Java}
\end{frame}

\begin{frame}[fragile]{Exemple} 
\begin{Java}
import java.util.Scanner;
public class Exemple {
  public static void main(String[] args) {
      Scanner clavier = new Scanner(System.in);
      int �ge;

      �ge = clavier.nextInt();
      if ( �ge<21 || �ge>=60 ) {
         System.out.println("Tarif r�duit !");
      }
  }
}
\end{Java}
\end{frame}

\begin{frame}[fragile]{Le \og selon-que\fg} 
Premi�re forme
\begin{Java}
  switch(produit) {
    case "Coca" : 
    case "Sprite" : 
    case "Fanta" :  
        prixDistributeur=60; 
        break;
    case "IceTea" : 
        prixDistributeur=70; 
        break;
    default :       
        prixDistributeur=0; 
        break;
  }
\end{Java}
\begin{itemize}
\item Notez le \java{break}
\item Possible avec : entiers, caract�res et chaines
\end{itemize}
\end{frame}

\begin{frame}[fragile]{Le \og selon-que\fg} 
Deuxi�me forme : la logique suivante
\begin{Code}
  Selon que 
    nb > 0 : �crire "positif"
    nb = 0 : �crire "nul"
    autres : �crire "n�gatif"
  Fin selon que
\end{Code}
s'�crit en Java
\begin{Java}
  if (nb>0) {
      System.out.println("positif");
  } else if (nb==0) {
      System.out.println("nul");
  } else {
      System.out.println("n�gatif");
  }  
\end{Java}
\end{frame}


\input{chapitre-erreur}







% ==S�ance 7
\section{La javadoc}
\leconwithtocquote{%
	\og  It's not a bug - it's an undocumented feature. \fg%
	\\Author Unknown}

% === Cours de Java
% === Chapitre : Javadoc
\subsection{La documentation Java}

\imgfullw{../img/explosion-paint.jpg}{
	\Large\bf\color{white}
	\vspace{-3.9cm}
	La documentation
}{}

\note{
	\par La documentation, une touche de couleur qui ne � sert � rien � \ldots
	mais qui change tout ! 
}

%%%
%\%subsubsection{Motivation}

\begin{frame}{Motivation}
	\begin{center}
		\Large\bf\color{bluepigment}
		\par Pour qui ?
		\par Qu'�crire ?
	\end{center}
\end{frame}

\note{
	\begin{itemize}
		\item Pour qui ?
		\begin{itemize}
			\item Le programmeur qui va \emph{utiliser} le code
			\item Le programmeur qui va \emph{maintenir} le code \\(peut-�tre vous)
		\end{itemize}
		\item Quel type de documentation ?
		\begin{itemize}
			\item \emph{Ce que fait} la m�thode/classe
			\item \emph{Comment} elle le fait
			\\(peut �tre r�duit au minimum si code lisible)
		\end{itemize}
	\end{itemize}
}

\note{
	Qui est int�ress� par quoi ?
	\begin{itemize}
		\item Le programmeur-\emph{utilisateur}
		\begin{itemize}
			\item int�ress� uniquement par le \emph{quoi}
		\end{itemize}
		\item Le programmeur-\emph{mainteneur}
		\begin{itemize}
			\item int�ress� par le \emph{quoi} et le \emph{comment}
		\end{itemize}
	\end{itemize}
}

\begin{frame}{Motivation}
	\textbf{O� mettre la documentation ?}
	\bigskip
	\begin{itemize}
		\item Avec le code
		\begin{itemize}
			\item Plus facile pour le maintenir
			\item Plus de chance de garder la synchronisation avec le code
		\end{itemize}
		\item Mais le programmeur-utilisateur n'a pas � voir le code pour l'utiliser
		\end{itemize}
\end{frame}

\begin{frame}{Motivation}
	\includegraphics[width=51px]{../img/icon_15706}
   	\emph{litterate programming}
   	\medskip
   	\begin{itemize}
		\item la documentation accompagne le code
		\item un outil extrait cette documentation pour en faire un document facile � lire
		\item toute la documentation suit la m�me structure, le m�me style
\\$\rightarrow$ plus facile � lire
	\end{itemize}
\end{frame}

\subsection{La Javadoc}

\imgfullh{../img/mur-cerisierjapon-9580504952_080b2591db_o.jpg}{
	\begin{center}
		\color{white}\Large\bf
		code $\xrightarrow{ javadoc  }$ doc
	\end{center}
}{}

\begin{frame}[fragile]{Javadoc}
	\sigle{\bf javadoc}
	\begin{itemize}
		\item Commentaire \sigle{javadoc} identifi� par  {\small \code|/** ... */|}
\begin{Java}
/**
    Calcule et retourne le maximum de 2 nombres.
*/
\end{Java}
		\item documentation produite au format \sigle{HTML}
		\item On commente essentiellement 
		\begin{itemize}
			\item la classe: r�le et fonctionnement
			\item les m�thodes publiques: ce que �a fait, param�tres et r�sultats
		\end{itemize}
		\item Se met \emph{juste au dessus} de ce qui est comment�
	\end{itemize}
\end{frame}

\subsection{Les tags}

\begin{frame}{Les tags}
	Utilisation de \emph{tags} pour identifier certains �l�ments
	\\\bigskip 
	Les plus courants :
	\begin{itemize}
		\item \emph{@param} : d�crit les param�tres 
		\item \emph{@return} : d�crit ce qui est retourn�
		\item \emph{@throws} : sp�cifie les exceptions lanc�es 
		\item \emph{@author} : note sur l'auteur
	\end{itemize}
\end{frame}

\begin{frame}[fragile]{Les tags}
\emph{Exemple}
\begin{Java}
/**
 * Donne la racine carr�e d'un nombre.
 * @param nb le nombre dont on veut la racine carr�e.
 * @return la racine carr�e du nombre.
 * @throws IllegalArgumentException si le nombre est n�gatif.
*/
public static double sqrt( double nb ) {...}
\end{Java}
\begin{itemize}
\item Les types sont d�duits de la signature et ajout�s � la documentation
\item La premi�re phrase (termin�e par un \java|.|) sert de r�sum�
\end{itemize}
\end{frame}

\begin{frame}{Les tags}
R�sum�
\vspace{-10pt}
\begin{center}
	\includegraphics[scale=.6]{../img/javadoc-resume}
\end{center}
\vspace{-10pt}
D�tail
\vspace{-10pt}
\begin{center}
	\includegraphics[scale=.6]{../img/javadoc-detail}
\end{center}
\end{frame}

%\subsubsection{Le code HTML}

\begin{frame}[fragile]{Le code HTML}
	Peut contenir des balises \sigle{HTML}
	\\\medskip
\emph{Exemple} :
\begin{Java}[basicstyle=\scriptsize]
  /**
  * Indique si l'ann�e est bissextile. Pour rappel :
  * <ul>
  *   <li>Une ann�e qui n'est pas divisible par 4 n'est pas bissextile 
  *       (ex: 2009)</li>
  *   <li>Une ann�e qui est divisible par 4</li>
  *     <ul>
  *     <li>est en g�n�ral bissextile (ex: 2008)</li>
  *     <li>sauf si c'est un multiple de 100 mais pas de 400 (ex: 1900, 2100)</li>
  *     <li>les multiples de 400 sont donc bien bissextiles (ex: 2000, 2400)</li>
  *     </ul>
  *   </ul>
  * Plus formellement, <code>a</code> est bissextile si et seulement si <br/>
  * <code>a MOD 400 = 0 OU (a MOD 4 = 0 ET a MOD 100 != 0)</code>
  * @param ann�e l'ann�e dont on se demande si elle est bissextile 
  * @return vrai si l'ann�e est bissextile
  */
\end{Java}
\end{frame}

\begin{frame}{Le code HTML}
	Ce qui donne
\begin{center}
	\includegraphics[scale=.7]{../img/javadoc-html}
\end{center}
\end{frame}

%%%
%\subsubsection{Produire la documentation}

\begin{frame}[fragile]{Production de la documentation}
	Commande \code|javadoc|
	\begin{itemize}
		\item Cr�ation de la documentation d'une classe 
		\\\java|javadoc Temps.java|
		\item Possibilit� de sp�cifier des sources multiples 
		\\\java|javadoc *.java|
		\item Cr�er la documentation dans un dossier sp�cifique 
		\\\java|javadoc -d doc *.java |
		\item Il y a beaucoup d'autres options \dots
		\\(cf. la documentation de \sigle{javadoc})
	\end{itemize}
\end{frame}

%%%
%\subsubsection{Pour une <<bonne>> documentation}

\begin{frame}{Une bonne documentation}
	Une bonne \textit{javadoc} d�crit le \emph{quoi} mais jamais le \emph{comment}
	\begin{itemize}
	\item $\longrightarrow$ Ne jamais parler de ce qui est priv�
	\item Mauvais exemples :
	\begin{itemize}
		\item \textit{On utilise un for pour parcourir le tableau.}
  		\item \textit{Pour aller plus vite, on stocke le prix hors tva dans 
  			une variable temporaire.}
  	\end{itemize}
	\end{itemize}
\end{frame}

\begin{frame}[fragile]{Une bonne documentation}
	Ne pas �crire ce que \sigle{javadoc} �crit lui-m�me :
	\begin{itemize}
		\item Mauvais exemples :
  		\begin{itemize}
  			\item \textit{nb - un entier qui ...}
  			\item \textit{La m�thode sqrt ...}
  			\item \textit{Cette m�thode ne retourne rien.}
  		\end{itemize}
	\item Pour en savoir plus :
{\scriptsize \code|http://www.oracle.com/technetwork/java/javase/documentation/index-137868.html|}
\end{itemize}
\end{frame}


% ==S�ance 8
\section{Les boucles}
\leconwithtoc
% === Cours de Java
% === Chapitre : Survol des instructions r�p�titives

\subsection{Les boucles (survol)}

\imgfullw{../img/looptheloop}
	{
		\begin{flushright}
			\color{aliceblue}\Large\it
			\vspace{-3cm}
			Loop the loop !\hspace{1cm}
		\end{flushright}
	}
	{https://www.flickr.com/photos/sirispjelkavik/2801926735/sizes/o/}

\begin{frame}[fragile]{Instructions r�p�titives}
Le \emph{Tant que} : 
\begin{Java}
  while ( condition ) {
    instructions
  }
\end{Java} 
\bigskip
\emph{Exemple} :
\begin{Java}
  int puissance = 1;
  while ( puissance < 1000 ) {
    System.out.println(puissance);
    puissance = 2 * puissance;
  }
\end{Java}
\end{frame}

\begin{frame}[fragile]{Exemple} 
\begin{Java}
import java.util.Scanner;
public class Exemple {
  /**
   * Affiche la somme d'entiers positifs entr�s au clavier.
   * S'arr�te d�s qu'une valeur nulle ou n�gative est donn�e.
   * @param args non utilis�
   */
  public static void main(String[] args) {
      Scanner clavier = new Scanner(System.in);
      int nb;
      int somme = 0;
      nb = clavier.nextInt();
      while ( nb > 0 ) {
         somme = somme + nb;
         nb = clavier.nextInt();
      }
      System.out.println(somme);
  }
}
\end{Java}
\end{frame}

\begin{frame}[fragile]{Instructions r�p�titives}
Le \emph{Pour} : 
\begin{Java}
  for ( int i=d�but; i<=fin; i=i+pas ) {
    instructions
  }
\end{Java} 
\bigskip
\emph{Exemple} :
\begin{Java}
  for ( int i=1; i<=10; i=i+1 ) {
    System.out.println(i);
  }
\end{Java}
\end{frame}

\begin{frame}[fragile]{Exemple} 
\begin{Java}
public class Exemple {
  /**
   * Affiche la somme des nombres pairs entre 2 et 100.
   * @param args non utilis�
   */
  public static void main(String[] args) {
      int somme;

      somme = 0;
      for ( int i=2; i<=100; i=i+2 ) {
         somme = somme + i;
      }
      System.out.println(somme);
  }
}
\end{Java}
\end{frame}

\begin{frame}[fragile]{Exemple} 
\begin{Java}
public class Exemple {
  /**
   * Affiche un compte � rebours � partir de 10.
   * @param args non utilis�
   */
  public static void main(String[] args) {

      for ( int i=10; i>=1; i=i-1 ) {
         System.out.println(i);
      }
      System.out.println("Partez !");
  }
}
\end{Java}
\end{frame}

\begin{frame}[fragile]{Instructions r�p�titives}
\java|i++| est un raccourci pour \java|i=i+1|
\\\bigskip
\emph{Exemple} :
\begin{Java}
  for ( int i=1; i<=n; i++ ) {
    System.out.println(i);
  }
\end{Java} 
\end{frame}

\full[bluepigment]{
	\begin{center}
		\Large\bf\color{azuremist}
		�tude de cas\\
		\it
		Lecture d'une donn�e enti�re positive
	\end{center}
}

\begin{frame}[fragile]{�tude de cas} 
\begin{itemize}
\item \emph{�tape 1} : lire un entier
\end{itemize}
\begin{Java}
/**
  * Lit un entier au clavier.
  * Les valeurs non enti�res sont pass�es.
  * @return l'entier lu.
  */
public static int lireEntier() {
    Scanner clavier = new Scanner(System.in);
    int nb;
    // Tant que ce n'est pas un entier au clavier
    while ( !clavier.hasNextInt() ) {
        clavier.next(); // le lire, le passer
    }
    nb = clavier.nextInt();
    return nb;
}
\end{Java}
\end{frame}

\begin{frame}[fragile]{�tude de cas} 
\begin{itemize}
\item \emph{�tape 2} : lire un entier positif
\end{itemize}
\begin{Java}
/**
  * Lit un entier au clavier.
  * Les valeurs non enti�res, nulles ou n�gatives sont pass�es.
  * @return l'entier lu.
  */
public static int lirePositif() {
    int nb;
    nb = lireEntier();
    while (nb<=0) {
      nb = lireEntier();
    }
    return nb;
}
\end{Java}
\end{frame}

\begin{frame}[fragile]{�tude de cas}
	\begin{itemize}
		\item \emph{�tape 3}: un exemple de main
	\end{itemize}
	\begin{Java}
/**
 * Un exemple de main.
 */
 public static void main(String[] args){
	int nombreLu;

	System.out.print("Entre un entier positif: ");
	nombreLu = lirePositif();
 }
	\end{Java}
\end{frame}






% ==S�ance 9
\section{Variables locales}
\leconwithtoc

% === Cours de Java
% === Chapitre : Les donn�es
%%%
%\section{Variables locales}

%%%
%\subsection{Pr�sentation}

\imgfullh{../img/Phangan__s_bucket__by_tiboudmiss.jpg}{
	\color{white}\bf\Large
	\vspace{-1.5cm}
	Variable locale
}{http://tiboudmiss.deviantart.com/art/Phangan-s-bucket-163211431}

\begin{frame}{Pr�sentation}
	D�signation g�n�rique d'un \emph{emplacement} de la \emph{m�moire vive}
	\begin{itemize}
		\item \emph{Poss�de un type}
		\item Ne peut contenir que des valeurs de ce type
		\item Allocation diff�rente si type \\\emph{primitif} ou \emph{r�f�rence}
	\end{itemize}
\end{frame}

%%%
\subsection{Allocation m�moire}

\begin{frame}{Allocation m�moire}
	Pour un type \emph{primitif}
	\begin{itemize}
		\item Indique la zone m�moire (sur la pile/{\it stack}) 
			o� se trouve la \emph{valeur}
	\end{itemize}
	\vspace{-1ex}
	\begin{center}
		\includegraphics[scale=.4]{../img/java-data-primitive} 
	\end{center}
\end{frame}

\begin{frame}{Allocation m�moire}
	Pour un type \emph{r�f�rence} (ex: \java|String|, tableau)
	\begin{itemize}
		\item La zone m�moire contient l'\emph{adresse} de la zone m�moire 
			(sur le tas/{\it heap}) contenant la valeur (indirection)
	\end{itemize} 
	\begin{center}
		\includegraphics[scale=.4]{../img/java-data-reference} 
	\end{center}
\end{frame}

%%%
\subsection{D�claration}

\begin{frame}[fragile]{D�claration}
	Une variable d�clar�e dans un \emph{bloc} \ldots 
	\\\hspace{1cm}est locale � ce \emph{bloc}		
	\bigskip
	\begin{grammaire}
	\nterm{Block} :
    \term{\{} \nterm{BlockStatements}\opt \term{\}}

	\nterm{BlockStatement} :
    \nterm{LocalVariableDeclarationStatement}
    \nterm{Statement}
	\end{grammaire}
\end{frame}

\begin{frame}[fragile]{D�claration}
	\textbf{D�claration}
	\par 
	\verb$<type> <identifier> [= expression]$
	\bigskip
	\par
	\emph{Exemples} :
	\begin{itemize}
		\item \java|int i;|
		\item \java|String nom, pr�nom;|
		\item \java|boolean ok=true, fini;|
		\item \java|char lettre, chiffre='1';| 
	\end{itemize}
\end{frame}


%%%
\subsection{Conventions sur les noms}

\begin{frame}{Nom d'une variable}
	\textbf{\textit{Identifier}}
	\par\bigskip
	Quel nom peut-on choisir ? 
	\begin{itemize}
  		\item R�gles \sigle{Java}
			\\\textit{javaletter}, \$,\_, \sout{1}
  		\item Conventions 
			\\\textit{mixedCase}, noms explicites
  	\end{itemize}
	\begin{flushright}
		\LARGE\bf\color{brickred} \$ \_ \,
	\end{flushright}
\end{frame}

\note{
	R�gles \emph{impos�es} par la grammaire
	\begin{itemize}
		\item Longueur illimit�e
		\item Compos� de \textit{lettres}, de \textit{chiffres}, 
			\java|$| et \java|_| %$
		\\(internationalisation)
		\item Ne commence pas par un chiffre
		\item $\neq$ \nterm{keyword} ou \nterm{litteral}
		\item Ex valides : \java|nom|, \java|Nom|, \java|Nom23|, 
			\java|Unpeu2touT|%$
		\item Ex invalides : \java|2main|, \java|le total|, \java|for|, 
			\java|true|, \java|12|
	\end{itemize}	
}

\note{
	Conventions \emph{suppl�mentaires}
	\begin{itemize}
		\item Utilis�es dans le monde entier
		\begin{itemize}
			\item Eviter \java|$| et \java|_| %$
			\item Commence par une minuscule
			\item Plusieurs mots accoll�s $\Rightarrow$ les suivants commencent par une majuscule ({\it mixedcase})
			\item Noms explicites (sauf abr�viations courantes)
			\item Articles omis
		\end{itemize} 
		\item Autres recommandations de \sout{\sigle{Sun}} \sigle{Oracle}
		\begin{itemize}
			\item D�clarer en d�but de bloc
			\item Une d�claration par ligne
		\end{itemize} 
	\end{itemize}
}


%%%
\subsection{Valeur initiale}

\begin{frame}[fragile]{Valeur initiale}
	\textbf{Initialisation}
	\par\medskip
	\begin{itemize}
		\item Par d�faut les variables \emph{ne} sont \emph{pas} initialis�es
		\item Initialisation avec n'importe quelle expression calculable � cet 
			endroit l� (� l'ex�cution)
	\end{itemize}
\end{frame}


\begin{frame}[fragile]{Valeur initiale}
	\textbf{Exemple}
	\begin{Java}
int poidsKilo = 20; // Un poids en kilos
int poidsGramme = 1000*poidsKilo; // L'�quivalent en grammes
	\end{Java}  
	\medskip
	\begin{Java}
int poidsKilo; // Un poids en kilos
int poidsGramme = 1000*poidsKilo; // Erreur � la COMPILATION
	\end{Java}  
\end{frame}

%%%
\subsection{Le concept de <<port�e>>}

\begin{frame}{Scope (port�e) d'une variable locale}
	\textbf{D�finition} Le \emph{\textit{scope}} (\emph{port�e}) d'une 
	variable indique la portion du programme o� elle \textit{existe}
	\par\bigskip
	Le \textit{scope} d'une variable locale est
	\begin{itemize}
		\item le \textit{block} de sa d�claration (entre \java|\{\}|)
		\item d�s sa propre initialisation
	\end{itemize} 
\end{frame}

\begin{frame}[fragile]{Scope (port�e) d'une variable locale}
	\emph{Exemples} : bon ou pas ?
\begin{Java}
{
   int x = y; 
   int y = 1;
}
\end{Java}
\begin{Java}
{
   int x;
   int y = x; 
}
\end{Java}
\begin{Java}
{
  int x = 1, y = x; 
}
\end{Java}
\end{frame}

%%%
\subsection{Constantes}

\begin{frame}[fragile]{Constante}
	\par
	{\large \java|final|}
	\par\bigskip
	Valeur donn�e
	\begin{itemize}
		\item Soit � la d�claration 
		\item Soit par assignation ult�rieure
	\end{itemize} 
	\bigskip
	\begin{Java}
  final int X = 1;
  final int Y;
  Y = 2*X;
  X = 2; // Erreur : poss�de d�j� une valeur
  Y = 3; // Idem
	\end{Java}
\end{frame}

\begin{frame}[fragile]{Constante}
	Convention de nom diff�rente
	\begin{itemize}
		\item Tout mettre en \emph{majuscules}
		\item Utiliser \java|_|  pour s�parer les mots
	\end{itemize} 
	\bigskip
	\emph{Exemples}
	\begin{Java}
	final double PI = 3.1415;
	final int TAUX_TVA = 21;
	\end{Java}
\end{frame}






%\include{chapitre-gram}
%\include{chapitre-lex}

%% === Cours de Java
% === Chapitre : Les expressions

%\section{Les expressions}
\subsection{Les expressions}

\imgfullh{../img/expression_by_bogdanboev-d5rs94g.jpg}{
	\Large\bf\color{white}
	Expression
}{http://bogdanboev.deviantart.com/art/Expression-348998704}

\full[bluepigment]{
	\color{azuremist}
	\textbf{Expression}
	\par\medskip
	Calcul faisant intervenir une ou plusieurs valeur(s) pour une op�ration 
	d�termin�e. 
	\\Cette valeur peut, elle-m�me �tre une expression.
}

\begin{frame}[fragile]{D�finitions}
	\emph{Exemple} : \java{1+2}
  	\begin{itemize}
		\item \java{1} et \java{2} sont les \textbf{op�randes}
		\item \java{+} est l'\textbf{op�rateur}
  		\item l'expression est de type \java{int}
  		\item la valeur de l'expression est $3$
  	\end{itemize}
\end{frame}


%%%
%\subsection{Les expressions enti�res}

\begin{frame}[fragile]{Les expressions enti�res}
	\emph{Op�rateurs entiers}
	\begin{center}
		\Large
		\java{+} \java{-}
		\hspace{5mm}
		\java{+} \java{-} \java{*} \java{/} \java{\%}
	\end{center}
	\medskip
	\emph{Op�randes} pouvant intervenir
	\begin{itemize}
		\item un \emph{litt�ral} 
		\item une \emph{variable}
		\item une \emph{expression}
	\end{itemize}
\end{frame}

\note{
	\par Distinguer les op�rateurs binaires et les unaires 
	\par Donner des exemples (beaucoup) d'expressions enti�res
}


\begin{frame}[fragile]{Les expressions enti�res}
	Les op�randes doivent �tre de \emph{m�me} type (entier)
	\\\textit{� conversion pr�s}
	\begin{itemize}	
		\item Le type de l'expression est celui de ses op�randes
		\item Exemples :
		\begin{itemize}
  			\item \java{1 + 2} vaut $3$ de type \java{int}
  			\item \java{1L + 2L} vaut $3$ de type \java{long}
  			\item \java{3 / 2} vaut $1$ de type \java{int}
  		\end{itemize}
	\end{itemize}
\end{frame}

\full[bluepigment]{
	\color{azuremist}
	\begin{center}
		\huge
		\texttt{i+i*2}
	\end{center}
}

\note{
	\par (i+i)*2 ou i+(i*2) ? 
	\par Notions de priorit� des op�rateurs
	\par En gros: � comme en math �
}

\begin{frame}[fragile]{Priorit� et associativit�}
	La strat�gie d'�valuation se base sur
  	\begin{itemize}
  		\item la \emph{priorit�} d'un op�rateur
  		\item l'\emph{associativit�} des op�rateurs de m�me priorit�
  	\end{itemize} 
\begin{center}
\begin{tabular}{r|c|c}
 priorit� & op�rateur & associativit� \\ \hline
grande & \java{-}, \java{+} unaires & $\Longleftarrow$ \\
 & \java{*}, \java{/}, \java{\%} & $\Longrightarrow$ \\
faible & \java{-}, \java{+} binaires & $\Longrightarrow$ \\
\end{tabular}
\end{center}
\end{frame}

\begin{frame}[fragile]{Priorit� et associativit�}
	\emph{Exercices}
  	\begin{itemize}
  		\item \java{3 + 3 * 2 + 1} ?
  		\item \java{3 + 3 * 2 / - 4 + 5 \% 8} ?
  	\end{itemize}  
	\bigskip
	\bigskip
	Les \emph{parenth�ses} permettent de \textbf{forcer} la strat�gie, 
	\textbf{expliciter} l'ordre
	et \textbf{clarifier}.
\end{frame}

\begin{frame}{Erreurs de calcul}
	La \emph{division par z�ro} 
	\begin{itemize}
		\item Lance une \textit{exception} (\textit{ArithmeticException})
		\item Pour l'instant, \emph{arr�te le programme} avec un message explicite
	\end{itemize} 
	\bigskip
	Le \emph{d�passement de capacit�} 
	\begin{itemize}
		\item N'est \emph{pas d�tect�} par la machine virtuelle
		\item Le r�sultat est tout simplement faux
	\end{itemize} 
\end{frame}

%%%
%\subsection{Les expressions flottantes}

\begin{frame}[fragile]{Les expressions flottantes}
	\emph{Op�rateurs flottants}
	\begin{center}
		\Large
		\java{+} \java{-}
		\hspace{5mm}
		\java{+} \java{-} \java{*} \java{/} 
	\end{center}
	\bigskip
	\emph{Op�randes} (litt�ral, variable ou expression) du m�me type 
	\emph{flottant} (\textit{� conversion pr�s})
\end{frame}

\begin{frame}[fragile]{Les expressions flottantes}
	\emph{Exemples} : Donner la valeur et le type de 
	\begin{itemize}
		\item \java{1.0 + 2.3}
		\item \java{1.0d - 2.3d}
		\item \java{7.0f / 3.5f}  
		\item \java{1. / 3.}
	\end{itemize} 
\end{frame}

%%%
%\subsection{Les expressions caract�res}

\begin{frame}[fragile]{Les expressions caract�res}
	\java{char} est un type num�rique entier
	\bigskip
	\begin{itemize}
		\item \emph{aucun op�rateur} sp�cifique 
		\item Calculs possibles mais non recommand�
	  	\\\emph{Exemple} : \java{'a' + 'b'}
	\end{itemize}
\end{frame}

%%%
%\subsection{Les chaines de caract�res}

\begin{frame}[fragile]{Les chaines de caract�res}
	\emph{Op�rateur}
	\begin{center}
		\Huge
		\java{+} 
	\end{center}
	\bigskip
	\begin{itemize}
		\item Un seul op�rateur pour la \emph{concat�nation} de deux chaines.
		\item Conversion si un des 2 op�randes n'est pas une chaine.
	\end{itemize}
\end{frame}

\note{
	\par Donner des exemples.
	\par Donner des exemples o� il y a conversion et o� la priorit� � 
	de l'importance
}

%%%
%\subsection{Les expressions bool�ennes}

\begin{frame}[fragile]{Les expressions bool�ennes}
	\emph{Op�rateurs bool�ens}
	\begin{center}
		\Large
		\Colorbox{listingback}{\verb|!|}  
		\hspace{5mm}
		\java{&&} \java{||}
	\end{center}
	\bigskip
	Tables de v�rit�
	\begin{center}
	\begin{tabular}{c|c|c|} 
	(ET) & \java{true} & \java{false} \\ \hline
	\java{true} & \java{true} & \java{false} \\ \hline
	\java{false} & \java{false} & \java{false} \\ \hline
	\end{tabular} 
	\hspace{4mm}
	\begin{tabular}{c|c|c|} 
	(OU) & \java{true} & \java{false} \\ \hline
	\java{true} & \java{true} & \java{true} \\ \hline
	\java{false} & \java{true} & \java{false} \\ \hline
	\end{tabular} 
\end{center}
\end{frame}

\note{
	\par Particularit� du ET: si l'op�rande de gauche est \emph{faux}, 
	l'op�rande droit \emph{ne sera pas �valu�} et le r�sultat sera \java{false}
	\par Particularit� du OU: si l'op�rande de gauche est \emph{vrai}, 
	l'op�rande droit \emph{ne sera pas �valu�} et le r�sultat sera \java{true} 
}

\begin{frame}[fragile]{Les expressions bool�ennes}
	\textbf{Tableau des priorit�s \#2}
	\begin{center}
	\begin{tabular}{r|c|c}
 	priorit� & op�rateur & associativit� \\ \hline
	grande & \java{-}, \java{+} unaires, \Colorbox{listingback}{\verb|!|} & $\Longleftarrow$ \\
	& \java{*}, \java{/}, \java{\%} & $\Longrightarrow$ \\
 	& \java{-}, \java{+} binaires & $\Longrightarrow$ \\
 	& \java{&&} & $\Longrightarrow$ \\
	faible & \java{||} & $\Longrightarrow$ \\
	\end{tabular}
	\end{center}
\end{frame}


\begin{frame}[fragile]{Exemples}
	Comment �valuer ?
	\begin{itemize}
		\item \java{true && false || true} 
		\item \Colorbox{listingback}{\verb/false || false && ! true/}
		\item \java{true || false && true} 
		\item \java{false && (true || false)} 
		\item \Colorbox{listingback}{\verb/!(true || false) && true/}
	\end{itemize} 
\end{frame}



%%%
%\subsection{Les expressions relationnelles}

\begin{frame}[fragile]{Les expressions relationnelles}
	\emph{Op�rateurs de comparaison}
	\begin{center}
		\Large
		\java{<}\quad \java{>}\quad \java{<=}\quad \java{>=}
	\end{center}
	\medskip
	\begin{itemize}
		\item Op�randes de type \emph{num�riques}
		\item Le \emph{r�sultat} est du type \emph{boolean}
	\end{itemize}
\end{frame}

\begin{frame}[fragile]{Les expressions relationnelles}
	\emph{Op�rateurs d'�galit�}
	\begin{center}
		\Large
		\java{==}\quad \Colorbox{listingback}{\verb|!=|}
	\end{center}
	\medskip
	\begin{itemize}
		\item Op�randes de \emph{tout type}
		\item Sens diff�rent si type primitif ou r�f�rence
	\end{itemize}
\end{frame}


\begin{frame}[fragile]{�galit� de valeurs}
	\textbf{Type primitif}
	\\Les \emph{valeurs} sont compar�es
	\begin{center}
	\includegraphics[scale=.5]{../img/java-logi-egal1} 
	\end{center}
	\bigskip
	On a : \java{a==c} mais \Colorbox{listingback}{\verb|a!=b|} et \Colorbox{listingback}{\verb|b!=c|}
\end{frame}

\begin{frame}[fragile]{�galit� de valeurs}
	\textbf{Type r�f�rence}
	\\Les \emph{r�f�rences} sont compar�es
	\begin{center}
	\includegraphics[scale=.5]{../img/java-logi-egal2} 
	\end{center}
	\bigskip
	On a : \Colorbox{listingback}{\verb|a!=b|}, \Colorbox{listingback}{\verb|a!=c|} mais \java{a==d}
\end{frame}

\full[bluepigment]{
	\color{azuremist}
	\par\textbf{Cas particulier du type \texttt{\bf String}}
	\par\medskip
	Le compilateur r�utilise l'espace pour les \emph{litt�raux} 
	de type \texttt{String}
	\bigskip
	{\tt\small
	\\String s1 = "Hello";
    \\String s2 = "Hello";
  	\\String s3 = "Hel";
  	\\s3 = s3 + "lo";
	\\System.out.println(s1==s2); // Vrai
	\\System.out.println(s1==s3); // Faux
  	\\s2 = "Bye";
	\\System.out.println(s1==s2); // Faux
	}
}


\begin{frame}[fragile]{Egalit� de valeur}
	\textbf{equals()}
	\medskip
\begin{Java}
{
  String s1 = "Hello";
  String s2 = "Hello";
  String s3 = "Hel";
  s3 = s3 + "lo";
  System.out.println(s1.equals(s2)); // Vrai
  System.out.println(s1.equals(s3)); // Vrai
}
\end{Java}
	\begin{itemize}
		\item Ne teste pas que les r�f�rences sont identiques
		\item Mais bien que les \emph{valeurs} r�f�renc�es sont �gales
	\end{itemize}
\end{frame}

%%%
%\subsection{Les expressions conditionnelles}

\begin{frame}[fragile]{L'expression conditionnelle}
	\emph{Op�rateur conditionnel}
	\begin{center}
		\huge
		\Colorbox{listingback}{\verb|?:|} 
	\end{center}
	\begin{itemize}
		\item �quivalent du \emph{si-sinon} sous forme d'expression
		\item {\tt <condition> ? <expression> : <expression>}
	\end{itemize}
	\medskip
	\emph{Exemples}
	\begin{itemize}
	\item \Colorbox{listingback}{\verb|heure < 12 ? "bonjour" : "bonsoir"|}
	\item  
	\begin{Java}
int n = -4; 
int abs = n > 0 ? n : -n;
	\end{Java}
	\end{itemize}
\end{frame}

\begin{frame}[fragile]{Tableau des priorit�s et associativit�s}
	\textbf{Tableau des priorit�s \#3}
\begin{small}
\begin{center}
\begin{tabular}{r|c|c}
 priorit� & op�rateur & associativit� \\ \hline
grande & \java{-}, \java{+} unaires, \Colorbox{listingback}{\verb|!|} & $\Longleftarrow$ \\
 & \java{*}, \java{/}, \java{\%} & $\Longrightarrow$ \\
 & \java{-}, \java{+} binaires & $\Longrightarrow$ \\
 & \java{<}, \java{>}, \java{<=}, \java{>=} & $\Longrightarrow$ \\
 & \java{==}, \Colorbox{listingback}{\verb|!=|} & $\Longrightarrow$ \\
 & \java{&&} & $\Longrightarrow$ \\
 & \java{||} & $\Longrightarrow$ \\
faible & \Colorbox{listingback}{\verb|?:|} & $\Longleftarrow$ \\
\end{tabular}
\end{center}
\end{small}
\end{frame}


%%%
%\subsection{Un mot sur les conversions}

\imgfullh{../img/Trick_of_Nature-croped.jpg}{
	\begin{center}
	\Large\bf\color{white}
	\vspace{16mm}
	Peut-on m�langer les types ?
	\end{center}
}{http://moscowbeast.deviantart.com/art/Tricky-Nature-24390118}

\note{
	\par 
	Peut-on m�langer les types ?
	\begin{itemize}
		\item Normalement pas
		\item Accept� si pas de perte d'information
		\item \emph{Conversion} effectu�e \emph{automatiquement} par le compilateur
		\item Une le�on enti�re sera consacr�e � ce sujet
	\end{itemize}
}

\begin{frame}[fragile]{Un mot sur les conversions}
	\emph{Exemples}
	\begin{itemize}
		\item Calcul m�langeant les entiers et les r�els
	  	\begin{itemize}
  			\item Les entiers sont convertis en r�els
  			\item \emph{Ex} : \java{3.2/2} vaut $1.6$ de type \java{double}
  		\end{itemize}
		\item Assigner un entier � un r�el
  		\begin{itemize}
  			\item L'entier est converti en r�el
  			\item \emph{Ex} : \java{double d = 1; // d vaut 1.0}
  		\end{itemize}
		\item Assigner un r�el � un entier
  		\begin{itemize}
  			\item Refus� \dots\ sauf si demand� explicitement (\emph{casting})
  			\item \emph{Ex} : \java{int i = 1.2; // refus�}
  			\item \emph{Ex} : \java{int j = (int) 1.6; // j vaut 1}
  		\end{itemize}
	\end{itemize}
\end{frame}
  
%\include{chapitre-assignation} 
%\include{chapitre-instr}
%\include{chapitre-tableau}

% ===== Cr�dits =====
\section*{Annexe}
\begin{frame}{Cr�dits}
Ce document a �t� produit avec les outils suivants
\begin{itemize}
\item {\small Les distributions \sigle{\emph{Ubuntu}} et/ou \sigle{\emph{debian}} 
du syst�me d'exploitation \sigle{\emph{Linux}}}
\item {\small \sigle{\emph{LaTeX}}/\sigle{\emph{Beamer}} comme syst�me d'�dition}
\item {\small \sigle{\emph{Git}} et \sigle{\emph{GitHub}} pour la gestion des versions et le suivi des corrections}
\item {\small Les outils \sigle{\emph{make}}, \sigle{\emph{rubber}}, \sigle{\emph{pdfnup}}, \dots}
\end{itemize}
\medskip
\href{http://creativecommons.org/licenses/by-sa/2.0/be}
{\includegraphics[width=16mm]{../img/cc-by-nc-sa-80x15.png}}
\end{frame}

\end{document} 
 
