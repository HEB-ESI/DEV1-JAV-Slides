% === Cours de Java
% === Chapitre : Survol 

\subsection{Algorithmes s�quentiels (survol)}
\subsection{Structure g�n�rale d'un programme}

\imgfullw{../img/Life_in_a_frame_by_michaelpaleodimos.jpg}
{}
{http://michaelpaleodimos.deviantart.com/art/Life-in-a-frame-39596992}

\begin{frame}[fragile]{Structure g�n�rale du programme}
	\texttt{\small\$cat NomClasse.java}
\begin{Java}
  public class NomClasse {
    // insert code here
  }
\end{Java}
\bigskip
\bigskip
\emph{Attention} \sigle{Java} est sensible � la casse 
\end{frame}

\begin{frame}[fragile]{La m�thode principale}
	\texttt{\small\$cat NomClasse.java}
\begin{Java}
  public class NomClasse {
    public static void main(String[] args) {
      // insert code here     
    }
  }
\end{Java}
\end{frame}


\begin{frame}[fragile]{Les variables}
Les types disponibles
\begin{center}
\begin{tabular}{r|l}
En Logique & En  Java \\ \hline
Entier & \java|int| \\ 
R�el & \java|double| \\ 
Chaine & \java|String| \\ 
Caract�re & \java|char| \\ 
Bool�en & \java|boolean| \\  
\end{tabular} 
\end{center}
Exemple de d�claration
\begin{Java}
   int nb1;
\end{Java}
\end{frame}

\begin{frame}[fragile]{L'assignation et les calculs}
	L'assignation se fait via le symbole 
	\begin{center}
		\huge\java|=|
	\end{center}
\begin{Java}
   nb1 = 1;
\end{Java}
\bigskip
Op�rateurs :
\begin{center}
	\Large
    \java|+| \java|-| \java|*| \java|/| \java|%|  
\end{center}
\end{frame}

\begin{frame}[fragile]{Exemple}
	\texttt{\small\$cat Moyenne.java}
\begin{Java}
  public class Moyenne {
    public static void main(String[] args) {

      double nombre1;
      double nombre2;
      double moyenne;

      nombre1 = 34345;
      nombre2 = -3213213;
      moyenne = (nombre1 + nombre2) / 2;    
      System.out.println(moyenne); 
    }
  }
\end{Java}
\end{frame}

\begin{frame}[fragile]{Exemple}
	\texttt{\small\$cat Moyenne.java}
\begin{Java}
  public class Moyenne {
    public static void main(String[] args) {

      int nombre1 = 34345;
      int nombre2 = -321321;
      double moyenne;

      // division r�elle car un des 2 op�randes est r�el
      moyenne = (nombre1 + nombre2) / 2.0;    
      System.out.println("La moyenne est " + moyenne); 
    }
  }
\end{Java}
\end{frame}


\begin{frame}[fragile]{Lire au clavier}
	\par\textit{Les applications modernes pr�f�rent les lectures dans des champs de saisies.}
	\par Dans une console, voici une mani�re de faire
	\medskip
	\par\emph{Exemple}
  \begin{Java}
  import java.util.Scanner;
  // ...
  Scanner clavier = new Scanner(System.in);
  // ...
  nombre1 = clavier.nextInt();
  \end{Java}
\end{frame}

\begin{frame}[fragile]{Lire au clavier - Exemple}
	\texttt{\small\$cat Test.java}
\begin{Java}
import java.util.Scanner;

public class Test {
  public static void main(String[] args) {
      Scanner clavier = new Scanner(System.in);
      double nombre1;
      double nombre2;
      double moyenne;

      nombre1 = clavier.nextDouble();
      nombre2 = clavier.nextDouble();
      moyenne = (nombre1 + nombre2) / 2.0;
      System.out.println(moyenne);    
  }
}
\end{Java}
\end{frame}

\begin{frame}{Lire au clavier}
\begin{center}
\begin{tabular}{r|l}
Pour lire\dots & on �crit\dots \\ \hline
un entier & \java|nextInt()| \\ 
un r�el & \java|nextDouble()| \\ 
un bool�en & \java|nextBoolean()| \\ 
un mot & \java|next()|\\  
une ligne & \java|nextLine()|\\  
un caract�re & \java|next().charAt(0)| \\ 
\end{tabular} 
\end{center}
\end{frame}

\subsection{Constantes}

\begin{frame}[fragile]{Constante locale}
	Une \textbf{constante} s'�crit gr�ce � \java|final| 
\par\medskip
\par\emph{Exemple}
\begin{Java}
  final int X = 1;
  final int Y;
  Y = 2*X;
  X = 2; // Erreur : poss�de d�j� une valeur
  Y = 3; // Idem
\end{Java}
\end{frame}

\subsection{Conventions}

\imgfullh{../img/people_think_theyre_different_by_nanfe-d3bzl5i_mirror.jpg}
{\vspace{19mm}\LARGE\bf\color{brickred} Conventions d'�criture}
{http://nanfe.deviantart.com/art/people-think-theyre-different-201534678}

\subsection{Conventions de noms}

\begin{frame}[fragile]{Conventions de noms}
Pour une variable :
\begin{itemize}
  \item Tout mettre en minuscules
  \item Sauf les d�buts de noms compos�s en majuscule
\end{itemize}

Pour une constante :
\begin{itemize}
  \item Tout mettre en majuscules
  \item Utiliser \_ pour s�parer les mots
\end{itemize}

Dans tous les cas : �tre explicite
\end{frame}

\subsection{Commentaires}

\begin{frame}[fragile]{Le commentaire}
Plusieurs mani�res d'ajouter un commentaire 
  \begin{Java}
  // Commentaire sur une ligne
  /* Commentaire sur
     plusieurs lignes */
  \end{Java}
\end{frame}


